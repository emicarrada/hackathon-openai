\documentclass[border=10pt]{standalone}
\usepackage{tikz}
\usepackage{xcolor}
\usepackage{fontawesome5}
\usetikzlibrary{shapes.geometric, arrows.meta, positioning, shadows, decorations.pathmorphing}

\definecolor{evaluador}{RGB}{255,107,107}
\definecolor{generador}{RGB}{78,205,196}
\definecolor{validador}{RGB}{255,195,113}
\definecolor{usuario}{RGB}{162,155,254}
\definecolor{resultado}{RGB}{116,185,255}

\tikzset{
    nodo/.style={
        rectangle,
        rounded corners=5pt,
        minimum width=3cm,
        minimum height=1.2cm,
        text centered,
        draw=black,
        line width=1.5pt,
        drop shadow,
        font=\sffamily\bfseries
    },
    evaluador/.style={nodo, fill=evaluador!30},
    generador/.style={nodo, fill=generador!30},
    validador/.style={nodo, fill=validador!30},
    usuario/.style={nodo, fill=usuario!30, minimum width=4cm},
    resultado/.style={nodo, fill=resultado!30, minimum width=4cm},
    flecha/.style={
        -Stealth,
        line width=2pt,
        draw=black!70
    },
    etiqueta/.style={
        font=\sffamily\small,
        text width=3.5cm,
        align=center
    }
}

\begin{document}

\begin{tikzpicture}[node distance=2.5cm]

% ========== TÍTULO ==========
\node[font=\sffamily\Huge\bfseries, text=black] at (0, 10) {Smart Optimizer - Hackathon OpenAI 2025};
\node[font=\sffamily\Large, text=black!70] at (0, 9.2) {Sistema de Optimización Inteligente con 3 Nodos};

% ========== USUARIO ==========
\node[usuario] (input) at (0, 7) {
    \faUser~USUARIO\\
    \small Input
};

% ========== NODO 1: EVALUADOR (Brandon) ==========
\node[evaluador, below=of input] (evaluador) {
    \faBrain~EVALUADOR\\
    \small Brandon
};

\node[etiqueta, right=0.3cm of evaluador, text width=5cm, align=left] (eval_detail) {
    \textbf{Clasificación Heurística:}\\
    • 24 keywords alta\\
    • 12 keywords baja\\
    • 6 keywords media\\
    • Análisis regex\\
    • Sistema de puntos
};

\node[etiqueta, below=0.2cm of evaluador] (eval_output) {
    \textcolor{evaluador}{\textbf{Output:}}\\
    Complejidad: \texttt{baja|media|alta}\\
    Modelo: \texttt{gpt-4o-mini|gpt-4o}
};

% ========== NODO 2: GENERADOR (Israel) ==========
\node[generador, below=3.5cm of evaluador] (generador) {
    \faCode~GENERADOR\\
    \small Israel
};

\node[etiqueta, right=0.3cm of generador, text width=5cm, align=left] (gen_detail) {
    \textbf{Generación con Cache:}\\
    • Consulta memoria caché\\
    • Genera respuesta inicial\\
    • Refina con contexto\\
    • Guarda en caché\\
    • Tokens optimizados
};

\node[etiqueta, below=0.2cm of generador] (gen_output) {
    \textcolor{generador}{\textbf{Output:}}\\
    Respuesta refinada\\
    Tokens usados: \texttt{N}\\
    Cache: \texttt{hit|miss}
};

% ========== NODO 3: VALIDADOR (Cristopher) ==========
\node[validador, below=3.5cm of generador] (validador) {
    \faCheckCircle~VALIDADOR\\
    \small Cristopher (Hub)
};

\node[etiqueta, right=0.3cm of validador, text width=5cm, align=left] (val_detail) {
    \textbf{Validación con LLM-Juez:}\\
    • Calidad de respuesta (0-10)\\
    • Comparación con gpt-4.1\\
    • Cálculo de ahorro\\
    • Tokens vs baseline\\
    • ROI y eficiencia
};

\node[etiqueta, below=0.2cm of validador] (val_output) {
    \textcolor{validador}{\textbf{Output:}}\\
    Calidad: \texttt{8.5/10}\\
    Ahorro: \texttt{75\%}\\
    Recomendación: \texttt{Óptimo}
};

% ========== RESULTADO ==========
\node[resultado, below=3cm of validador] (output) {
    \faCheckDouble~RESULTADO\\
    \small Respuesta + Métricas
};

% ========== AUTOMEJORA (FEEDBACK LOOP) ==========
\node[draw=purple!70, very thick, rounded corners, fill=purple!10, text width=3.5cm, align=center, left=1cm of generador, yshift=-3cm] (automejora) {
    \textbf{\faRecycle~AUTOMEJORA}\\[0.2cm]
    \small Ajuste dinámico\\
    de estrategias
};

% ========== FLECHAS PRINCIPALES ==========
\draw[flecha] (input) -- node[etiqueta, right=0.2cm] {
    \small Tarea del usuario:\\
    \textit{"Explica ML"}
} (evaluador);

\draw[flecha] (evaluador) -- node[etiqueta, right=0.2cm] {
    \small Complejidad:\\
    \texttt{"media"} → \texttt{gpt-4o-mini}
} (generador);

\draw[flecha] (generador) -- node[etiqueta, right=0.2cm] {
    \small Respuesta generada:\\
    + tokens usados
} (validador);

\draw[flecha] (validador) -- node[etiqueta, right=0.2cm] {
    \small Validación completa:\\
    Calidad + Ahorro + ROI
} (output);

% ========== FLECHA DE FEEDBACK (AUTOMEJORA) ==========
\draw[flecha, purple!70, very thick, dashed, -Stealth] 
    (validador.west) 
    to[out=180, in=90] 
    node[etiqueta, left=0.1cm, text width=2.5cm] {
        \small \textcolor{purple}{\textbf{Feedback}}\\
        Si calidad < 7/10\\
        → Ajustar estrategia
    } 
    (automejora.north);

\draw[flecha, purple!70, very thick, dashed, -Stealth] 
    (automejora.north) 
    to[out=90, in=180] 
    node[etiqueta, left=0.1cm, text width=2.5cm] {
        \small \textcolor{purple}{\textbf{Reajuste}}\\
        Actualiza umbral\\
        de complejidad
    } 
    (evaluador.west);

% ========== MODELOS A LA IZQUIERDA ==========
\node[draw=black!50, thick, rounded corners, fill=black!5, text width=3.5cm, align=center, left=1cm of evaluador, yshift=1cm] (modelos) {
    \textbf{\faServer~Modelos OpenAI}\\[0.3cm]
    \textcolor{green!60!black}{\texttt{gpt-4o-mini}}\\
    \small Complejidad baja/media\\[0.2cm]
    \textcolor{blue!60!black}{\texttt{gpt-4o}}\\
    \small Complejidad alta\\[0.2cm]
    \textcolor{red!60!black}{\texttt{gpt-4.1}}\\
    \small Baseline (comparación)
};

% ========== MÉTRICAS A LA DERECHA ==========
\node[draw=black!50, thick, rounded corners, fill=black!5, text width=3.5cm, align=center, right=6.5cm of validador, yshift=3cm] (metricas) {
    \textbf{\faChartLine~Métricas Clave}\\[0.3cm]
    \faCoins~\textbf{Ahorro de costos}\\
    \small 60-80\% vs baseline\\[0.2cm]
    \faClock~\textbf{Latencia}\\
    \small Optimizada con cache\\[0.2cm]
    \faAward~\textbf{Calidad}\\
    \small Validada por LLM-Juez\\[0.2cm]
    \faRocket~\textbf{ROI}\\
    \small Alto rendimiento
};

% ========== EQUIPO ABAJO ==========
\node[draw=black!50, thick, rounded corners, fill=black!5, text width=12cm, align=center, below=0.5cm of output] (equipo) {
    \textbf{\faUsers~Equipo Hackathon}\\[0.2cm]
    \textcolor{evaluador}{\textbf{Brandon}} (Evaluador) $\mid$ 
    \textcolor{generador}{\textbf{Israel}} (Generador) $\mid$ 
    \textcolor{validador}{\textbf{Cristopher}} (Hub/Validador)\\[0.1cm]
    \small 23 de octubre, 2025
};

% ========== LEYENDA TECNOLOGÍAS ==========
\node[draw=black!50, thick, rounded corners, fill=black!5, text width=3.5cm, align=left, left=1cm of validador, yshift=-1cm] (tech) {
    \textbf{\faCog~Stack Técnico}\\[0.2cm]
    • LangGraph 0.0.40+\\
    • OpenAI API\\
    • Python 3.10+\\
    • pytest (testing)\\
    • JSON (estrategias)
};

\end{tikzpicture}

\end{document}
